\documentclass[journal]{IEEEtran}
\usepackage{epsf,cite,amsmath,amscd,graphics,graphicx,latexsym,multicol,setspace}
\usepackage{amsfonts,amsmath,amssymb,amsthm}
\usepackage{scalefnt}
\newcounter{MYtempeqncnt}
\begin{document}
\title{Low-Dose CT Image denoising and Restoration using General Adversarial Networks}

\author{Kevin Freire $|$ kfreirea@ryerson.ca\\
Walter Freire $|$ andrei.freire@ryerson.ca \\
Department of Electrical and Computer Engineering\\
Ryerson University, Toronto, Canada.}
\maketitle


% ABSTRACT
\begin{abstract}
Medical Imaging has been a growing topic in computer vision for its life saving applications.  Medical professionals rely on good quality images of their scan in order to correctly identify tumours or other anomalies.  Recent studies have shown that Computerize Tomography CT scans have great results using high radiations for their X-Rays.  Studies have shown that these CT Scans have produced more than half the radiation from medical use which results in problems for long term use of these expensive machines.  Some solutions have involved reducing the X-Ray magnitude in order to reduce exposure to X-Ray but this results in lower quality images with a lot of noise.  We implemented a denoising neural network trained on medical images to reduce the noise from low dose CT Scan.  This model is capable of producing High Dose quality imaging using low dose CT Scans at a reasonable rate.


\end{abstract}

% KEYWORDS
\begin{IEEEkeywords}
Image restoration, Image Denoising, General Adversarial Networks, GANs, Low-Dose CT, Medical Imagery, Deep Neural Networks.
\end{IEEEkeywords}

% INTRODUCTION
\section{Introduction}
\label{Introduction}
Computerize Tomography (CT) has enabled direct imaging on the 3-dimensional structure of different organs and tissues inside the human body in a non-invasive manner. CT scans are constructed by combining the X-ray scans taken on several angles and orientations.  It has several utilities but very useful in detecting lesions, tensions, tumours, and metastasis.  It can reveal their presence and the spatial location, size and extent of the tumour.  CT imaging has become a frequent tool for cancer diagnosis, angiography, and detecting internal injuries.  However, despite the evidence of its utility for diagnosis and patient management, the potential risk of radiation-induced malignancy exists \cite{brenner2007computed}.  Studies found that CT alone contributes to almost half of the total radiation exposure from medical use alone.  Recent studies reveal that as much as 1.2 - 2\% of cancers may eventually be caused by the radiation dose conceived by the patient while undergoing CT scans \cite{schauer2009national}.  To reduce the risk, the principles of ALARA (As Low As Reasonably Achievable) is now a profound practice predicted in CT imaging \cite{protection2007icrp}.  In regards to this practice, Low Dose CT (LDCT) is a promising solution in reducing radiation exposure \cite{trattner2014standardization}.  In low dose CT, radiation exposure is decreased by lowering the tube current, or voltage.  However, by reducing the tube voltage or current it introduces several artefacts and lowers the diagnostic quality of the LDCT image \cite{boas2012ct}.  In order to boost the quality of an LDCT image, the reconstruction of LDCT has become a primary research.  There are various methods that can be classified under three categories: (a) iterative reconstruction, (b) sinogram filtration based techniques, and (c) image post-processing based technique.  In recent times, researchers were trying to develop new iterative algorithms (IR) for LDCT image reconstruction.  Iterative reconstruction algorithms considerably suppresses the image noise, but still lose some details and suffer from remaining artefacts.  Other disadvantages with IR techniques is the high computational cost, which is a bottle neck in practical utilization.  Sinogram filtration on the other hand, directly works on the projection data before reconstructing the image and is more computationally economical than the IR technique.  However, the data of commercial scanners are not readily available to users, and this method suffers from edge blurring and the resolution loss.  Many efforts were made in the image domain to reduce LDCT noise and suppress artefacts.\\
With the explosive evolution of deep neural networks, the LDCT denoising task is now dominated by deep neural networks.  However, the research on deep learning-based LDCT denoising is confined to designing a network architecture based on the vanilla convolution operation.  There has been a surge in interest in desinging General Adversarial Networks (GAN), mainly for image generation and becoming popular in denoising medical images as shown in \cite{8340157}, \cite{9474492}, \cite{yin2021unpaired}.  In this paper we explore the possibility of applying GAN to the task of LDCT desnoising.\\
	In many image-related reconstruction tasks, it is known that minimizing the per-pixel loss between the output image and the ground truth alone generate either blurring or makes the result visually unappealing \cite{huang2017beyond}.  The same effect was observed in the traditional neural network-based CT denoising works \cite{chen2017low}, \cite{chen2017low2}.  The adversarial loss introduced by GAN can be treated as a driving force to push the generated image to look as close to as the groung truth image or in this case the Normal Dose CT image (NDCT) which also reduces the blurring effect. Further more, an additional perceptual loss was also introduced to measure the feature of the denoised image, with a focus on areas that the human eye cannot see. \\
Generative Adversarial Network was first introduced in 2014 by Goodfellow \emph{et al}. \cite{goodfellow2014generative}.  It is a genrative model trying to generate real world images by employing a min-max optimization framework where two networks (Generator G and Discriminator D) are trained against each other.  G tries to synthesize real appearing images from random noise wheras D is trying to distiguish between the generated and real images.  If the Generator G get sufficiently well trained, the Discriminator D will eventually be unable to tell if the generated image is fake or not.\\
	The original setup of GAN does not contain any constraints to control what modes of data it can generate.  However, the auxiliary information were provided during the generation, GAN can be driven to output images with specific modes.  In this scenario, GAN is usually referred to as conditional GAN (cGAN) since the output is conditioned on additional information.  Bera \emph{et al}. \cite{9474492} proposed a discriminator function for CT denoising task, which leverages self-attention and pixel-wise GANs for restoring the diagnostic quality of LDCT images.  Yi \emph{et al}. \cite{yi2018sharpness}. propsed an adversarial trained network and a sharpness detection network to denoise LDCT images. Both \cite{8340157} and \cite{yin2021unpaired} used Wasserstein GAN with Perceptual Loss and Wasserstein Distance to generate a denoised image of an LDCT image. \\
	In this paper, we make the following contributions:
	
	\begin{enumerate}
		\item We built a GAN architecture similar to Alsaiari \emph{et al}. \cite{alsaiari2019image}. with the exception of trianing it pecifically on LDCT images.
		\item Computed similar loss functions as \cite{alsaiari2019image} with the exception of using the Mean Squared Error Loss function instead of computing L2-norm.
		\item Present the output results of the network after training.
	\end{enumerate}
	
	The remainder of this paper is organized as follows.  The proposed method based on \cite{alsaiari2019image}, with perceptual loss, pixel-pixel loss, smooth loss and adversarial loss are presented in \ref{method}.  Then, experiments and analysis of the results are presented in section \ref{experiment settings} and \ref{results and discussion}.  Lastly, Section \ref{conclusion} gives a summary of this paper and looks forward to some possible future research directions.

% METHOD
\section{Method}
\label{method}
	We built a GAN network structure for image denoising, which like, \cite{alsaiari2019image}, is based on ResNet \cite{he2016deep}.  A generator network is trained to generate noise fre images through comptition with the discriminator network, using ground truth NDCT images to imporve the quality of the generated images.  Our network makes use of residual blocks, skip connections, and batch normalization.  Due to training limitations, we used three residual blocks.  However, having a larger number of residual blocks would increase the training accuracy significantly, but with a higher computational complexity and longer training times.\\
	Suppose that $z \in \{z_k\}_1^N$ denotes the LDCT image, $x \in \{x_k\}_1^N$ denotes the NDCT image, and $p_l$ and $p_r$ represent the distribution of LDCT and NDCT images, respectively.  The generator G learns a mapping $G: z \rightarrow \hat{x}$, where \emph{z} is the LDCT the generator is conditioned upon.  $\hat{x}$ is the denoised CT that is expected to be as close to as the NDCT image \emph{x}.  The discriminator D, is to differentiate the generated image \emph{z} from the real \emph{x}.  
	
% LOSS FUNCTION
\subsection{Refined Loss Function}
\label{loss function}
In the original GAN \cite{goodfellow2014generative}, D and G are trained by solving the following minmax problem\\

\begin{equation}
	\mathop{min}_{G}\mathop{max}_{D} L_{GAN}(D,G) = -E_{x \tilde p_r}[D(x)] + E_{z \sim p_l}[1-D(G(z))]
\end{equation}

where $E(\cdot)$ denotes the expectation operator, $G(\cdot)$ and $D(\cdot)$ represents the outputs of G and D.  The generator G transforms a noisy sample to mimic a real sample, which defines a data distribution, denoted by $p_g$.  Then D is trained to become an optimal discriminator for a fixed G, the minimization search for G is equivalent to minimizing the Jensen-Shannon (JS) divergence of $p_r$ and $p_g$, which will lead to a vanished gradient on the Generator G  \cite{8340157} and G will stop updating as the training continues.  \\
	In the process of using CT image training, the feature difference between the LDCT and NCT image will have a certain probability of structural loss in denoising results.  In order to obtain a good quality image, we include the Minimum Square Error Loss (MSE) between the Generated image $\hat{x}$ and the ground truth $x$.  The function is expressed as follows:
	
\begin{equation}
	L_{MSE} = E_{z\sim p_l}\left[ \frac{1}{N}\|x - G(z)\|^2 \right],
\end{equation}
	
Where $\|\cdot\|^2$ is the square L2-norm (MSE Loss).  A lower difference denotes a higher quality of the generated image.  However, the MSE loss can potentially generate blurry images and cause the distortion or loss of details.  Thus, we also applied the perceptual loss.\\
	With perceptual loss, the rational behind it is two-fold.  First, when a person compares two images, the perception is not performed pixel-by-pixel.  The human vision extracts and compares features from images \cite{nixon2019feature}.  With perceptual loss, it guides the model towards image style conversion \cite{johnson2016perceptual}, image denoising \cite{8340157}, and other tasks.  The perceptual loss function is introduced to learn feature distribution of NDCT images from the feature space to guide the denoising task of LDCT images.  Therefore, we included a pre-trained deep CNN (VGG16) for feature extraction and compared the denoised output against the ground truth in terms of the extracted features.  The perceptual loss function is defined in a feature space:
	
\begin{equation}
	L_{Perceptual}(G) = E_{(x,z)}\left[ \frac{1}{whd}\|\Phi(G(z))-\Phi(x)\|^2 \right],
\end{equation}

	where $\Phi$ is the feature map obtained from the output of the second convolutional layer of the VGG-16 pretrained network, $G(z)$ is the denoising result of the LDCT image, \emph{w}, \emph{h}, \emph{d} represents width, height and depth of the feature map, respectively.  It should be noted that we used the VGG-16 network as the feature extractor.  The input of the network is a colour image, including three channels, while the CT images is greyscale image.  Thus, before suing the network for feature extraction, we duplicated the CT image to make RGB channels.

% GENERATOR NETWORK
\subsection{Generator Network}
\label{generator}
The goal of a single image denoising is to generate a realistic image with high quaility.  The 

% DISCRIMINATOR NETWORK
\subsection{Discriminator Network}
\label{discriminator}


% EXPERIMENT SETTINGS
% this section includes dataset and training details
\section{Experiment Settings}
\label{experiment settings}
The network was trained using Google Colab Pro which use a K80 GPU and 24 GB of RAM using the Pytorch framework. We used a batch size of 16,  and ran over 15000 iterations for training which was around 500 epochs.  Our Dataset was obtained from 2016 NIH-AAPM-Mayo Clinic Low Dose CT Grand Challenge by Mayo Clinic.  During training we set different hyper parameters which include $\lambda_a=0.5$, $\lambda_p=1.0$, $\lambda_f=1.0$ and $\lambda_s=0.0001$. We set $K=32$ and $k_2=48$ for the generator and discriminator channel filters.  For the convolutional and deconvolution layers of the generator we used a kernel size of 9 and stride 1.  All the other convolutional layers in the generator we had a kernel size of 3x3 and stride 1.   The discriminator first 3 layers, all of its convolutional networks had a 4x4 kernel size with stride 2 and zero padding by 1.  the last two layers in D are composed of kernels of size 4x4 with stride 1 and zero-padding of 1. To train the discriminator and generator we took the approach of training one at a time.  We did this by training the discriminator first for 2 epochs and then training the generator for 10 epoch, then the cycle repeats.  We have implemented our approach and version of our code can be obtained from https://github.com/kevinfreire/ldct-denoising.
% RESULTS AND DISCUSSIONS
\section{Results and Discussion}
\label{results and discussion}

% COMPUTATIONAL COMPLEXITY
\section{Computational Complexity}
\label{complexity}

% CONCLUSION
\section{Conclusion}
\label{conclusion}
The denoising network is an end-to-end operation, in which the input is a low dose CT Scans and the output is a high dose CT scan.  The results we obtained were not as comparable as the papers we have referenced due to small tweaking of the networks we have done, but the general concept still remains.  This network needs to be fine tuned as well as tested on other images that are not CT scans and see its performance.  We also would like to explore using newer VGG models and different Loss functions to observe results.  
% REFERENCES
\cleardoublepage
\cite{8340157} Yang \emph{et al} - Low-Dose CT Image Denoising Using a Generative Adversarial Network With Wasserstein Distance and Perceptual Loss

\cite{9474492} Bera \emph{et al} - Noise Conscious Training of Non Local Neural Network Powered by Self Attentive Spectral Normalized Markovian Patch GAN for Low Dose CT Denoising

\cite{yin2021unpaired} Yin \emph{et al} - Unpaired Image Denoising via Wasserstein GAN in Low-Dose CT Image with Multi-Perceptual Loss and Fidelity Loss

\cite{radford2015unsupervised} Radford \emph{et al}. - UNSUPERVISED REPRESENTATION LEARNING WITH DEEP CONVOLUTIONAL GENERATIVE ADVERSARIAL NETWORKS

\cite{chen2017low} Chen \emph{et al}. - Low-dose ct via deep neural network

\cite{chen2017low2} Chen \emph{et al} - Low-dose CT with a residual encoder-decoder convolutional neural network

\cite{huang2017beyond} Huang \emph{et al}. - Beyond face rotation: Global and local perception gan for photorealistic and identity preserving frontal view synthesis

\cite{goodfellow2014generative} Goodfellow \emph{et al}. - Generative adversarial nets

\cite{yi2018sharpness} Yi \emph{et al}. - Sharpness-aware low-dose CT denoising using conditional generative adversarial network



\bibliography{project_reference}
\bibliographystyle{ieeetr}
\end{document}